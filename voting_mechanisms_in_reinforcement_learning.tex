\documentclass[conference]{IEEEtran}
\usepackage{cite}
\usepackage{amsmath,amssymb,amsfonts}
\usepackage{algorithmic}
\usepackage{graphicx}
\usepackage{textcomp}
\usepackage{xcolor}

\def\BibTeX{{\rm B\kern-.05em{\sc i\kern-.025em b}\kern-.08em
    T\kern-.1667em\lower.7ex\hbox{E}\kern-.125emX}}
\begin{document}

\title{Voting Mechanisms in Reinfocement Learning}

\author{\IEEEauthorblockN{Jost Alemann}
\IEEEauthorblockA{\textit{Institute of intelligent \& cooperating systems} \\
\textit{Otto von Guericke University}\\
Magdeburg, Germany \\
jost.alemann@ovgu.de}
}

\maketitle

\begin{abstract}
This paper aims to deliver an overview over voting mechanisms used in reinforcement learning.
Voting mechanisms are first introduced to the reader and then explained in more detail by
describing usage examples and experiments from current research. 

\end{abstract}

\begin{IEEEkeywords}
component, formatting, style, styling, insert
\end{IEEEkeywords}

\section{Introduction}
In multi-agent settings there are a number of mechanisms that allow for interaction between several agents.
One of those mechanisms is voting which allows multiple agents to decide on a single action. 

\section{Related Work}

\bibliographystyle{IEEtran}
\bibliography{references}

\vspace{12pt}

\end{document}
