\documentclass[conference]{IEEEtran}
\usepackage{cite}
\usepackage{amsmath,amssymb,amsfonts}
\usepackage{algorithmic}
\usepackage{graphicx}
\usepackage{textcomp}
\usepackage{xcolor}

\def\BibTeX{{\rm B\kern-.05em{\sc i\kern-.025em b}\kern-.08em
    T\kern-.1667em\lower.7ex\hbox{E}\kern-.125emX}}
\begin{document}

\title{Voting Mechanisms in Reinfocement Learning}

\author{\IEEEauthorblockN{Jost Alemann}
\IEEEauthorblockA{\textit{Institute of intelligent \& cooperating systems} \\
\textit{Otto von Guericke University}\\
Magdeburg, Germany \\
jost.alemann@ovgu.de}
}

\maketitle

% Research questions:
    % How voting can be incorporated inside multi-agent reinforcement learning algorithms?
    % What are the benefits?
    % Which voting mechanisms are used?

%ABSTRACT
\begin{abstract}
This paper aims to deliver an overview over how voting mechanisms can be incorporated in reinforcement learning.
Voting mechanisms and their properties are first introduced to the reader and then explained in more detail by describing their application in related work in the field of multi-agent systems and reinforcement learning.
\end{abstract}

\begin{IEEEkeywords}
voting, reinforcement learning, multi-agent systems
\end{IEEEkeywords}

% INTRODUCTION
    % Motivation – Why is our research interesting
    % Problem – Where is a lack of knowledge
    % Idea – What solution do you present
    % Outline – How do you explain your insight
\section{Introduction}
In a democratic society choices are not made by a dictator but by taking the preferences of the whole society into account. Therefore each member of the society is entitled to cast a vote which represents it's preferences. Votes are then evaluated and a decision is derived by a given voting scheme. Plurality where the choice with the most votes wins, might be the most common voting scheme. Still there are many different voting schemes each of which can or cannot fulfil certain properties. This lies mainly in the subject of social choice theory and therefore is only described briefly if needed in the following.
\newline
The concept of considering multiple individuals' preferences by using a voting mechanism to make choices can be transferred to multi-agent reinforcement learning systems. This is motivated by the expectation that agents combine their limited perception and knowledge of the environment by deciding together. Therefore they are expected to obtain better results than agents that choose actions based on only their own perception.\cite{partalas2008hybrid}
\newline
Since learning agents try to optimise their own reward, they might learn to act selfish\cite{carr2008peer} or even learn to use strategic voting to exploit the voting system\cite{pitt2006voting} for that purpose.
Thus the design of a voting mechanism incorporated in a multi-agent reinforcement learning system is non-trivial.
To avoid extensive selfish behaviour in agents several constraints can be introduced by modifying the reward function or voting system.
The possibility to exploit a voting system depends on the chosen voting scheme.
Unfortunately Arrow's Impossibility Theorem implies that no voting scheme can be designed to be completely fair. This means there is always a way in which agents could exploit such a voting scheme by finding a certain voting strategy. Arrow's Impossibility Theorem will be explained later on in Section \ref{2BasicPrinciples}.
Ongoing research constantly tries to improve the security, robustness and fairness of voting mechanisms.
\newline
To give an overview over voting mechanisms in reinforcement learning Section \ref{2BasicPrinciples} introduces basic principles of social choice theory.
Section \ref{3RelatedWork} describes related work to highlight use cases of voting mechanisms in reinforcement learning as well as ongoing research trying to improve such mechanisms.

% BASIC PRINCIPLES
    % Tell the readers about definitions, foundations that are used
\section{Basic Priniciples}\label{2BasicPrinciples}

% Voting Schemes
\subsection{Different Voting Schemes}
To discuss properties of voting systems we have to introduce those systems first.

% Properties of Voting Schemes
\subsection{Properties of Voting Schemes}
\begin{itemize}
    \item \textit{Pareto efficiency}: If every voter prefers option X over option Y, then the society prefers X over Y.
    \item \textit{Independence of irrelevant alternatives}: If every voter prefers option X over option Y and option Z is removed without changing the former relation, the societies preference of X over Y also remains unchanged.
    \item \textit{Non-dictatorship}: No single voter posseses the power to always determine the group's preference.
\end{itemize}

% Arrow's Impossibility Theorem
\subsection{Arrow's Impossibility Theorem}
Arrow's Impossibility Theorem is of great importance for the design of a voting mechanisms. It states that no rank-order voting scheme can fulfil the properties Pareto efficiency, independece of irrelevant alternatives and non-dictatorship at the same time.

% RELATED WORK
    % How have other researchers tackled the problem
    % Outline benefits and downsides of other methods
    % Give a good structure to the current state of the art
\section{Related Work}\label{3RelatedWork}


\bibliographystyle{IEEEtran}
\bibliography{refs}

\vspace{12pt}

\end{document}
