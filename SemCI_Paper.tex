\documentclass[conference]{IEEEtran}
\usepackage{cite}
\usepackage{amsmath,amssymb,amsfonts}
\usepackage{algorithmic}
\usepackage{graphicx}
\usepackage{textcomp}
\usepackage{xcolor}

\def\BibTeX{{\rm B\kern-.05em{\sc i\kern-.025em b}\kern-.08em
    T\kern-.1667em\lower.7ex\hbox{E}\kern-.125emX}}
\begin{document}

\title{Voting Mechanisms in Reinfocement Learning}

\author{\IEEEauthorblockN{Jost Alemann}
\IEEEauthorblockA{\textit{Institute of intelligent \& cooperating systems} \\
\textit{Otto von Guericke University}\\
Magdeburg, Germany \\
jost.alemann@ovgu.de}
}

\maketitle

%ABSTRACT
\begin{abstract}
This paper aims to deliver an overview over voting mechanisms used in reinforcement learning.
Voting mechanisms are first introduced to the reader and then explained in more detail by
describing usage examples and experiments from current research. 
\end{abstract}

\begin{IEEEkeywords}
voting, reinforcement learning, multi-agent systems
\end{IEEEkeywords}

% INTRODUCTION
    % Motivation – Why is our research interesting
    % Problem – Where is a lack of knowledge
    % Idea – What solution do you present
    % Outline – How do you explain your insight
\section{Introduction}
In a democratic society choices are not made by a dictator but by taking the preferences of the whole society into account. Each member of the society is entitled to cast a vote which represents it's preferences.
This concept of considering multiple individuals' preferences by using a voting mechanism to make choices can be transferred to multi-agent reinforcement learning systems.
\newline
However designing a voting mechanism is non-trivial since the voting system has to consider different aspects like security, fairness and robustness.
Arrow's Impossibility Theorem even states that voting mechanisms cannot be designed to be completely fair.
Ongoing research constantly tries to improve the security, robustness and fairness of voting mechanisms.
\newline
To give an overview over voting mechanisms in reinforcement learning Section \ref{2BasicPrinciples} introduces basic principles and constraints of voting systems.
Section \ref{3RelatedWork} describes related work to highlight use cases of voting mechanisms in reinforcement learning as well as ongoing research trying to improve such mechanisms.

% BASIC PRINCIPLES
    % Tell the readers about definitions, foundations that are used
\section{Basic Priniciples}\label{2BasicPrinciples}

% Arrow's Impossibility Theorem
\subsection{Arrow's Impossibility Theorem}
Arrow's Impossibility Theorem is of great importance for the design of a voting mechanism. 

\begin{itemize}
    \item Robert's Rules of Order (Newly Revised) (RONR)
    \begin{itemize}
        \item Set of rules for voting procedure
    \end{itemize}
    \item Arrow's Impossibility Theorem
    \begin{itemize}
        \item states that no rank-order electorial system can be designed to be perfectly fair matching these criterias:
        \item If every voter prefers alternative X over alternative Y, then the group prefers X over Y.
        \item If every voter’s preference between X and Y remains unchanged, then the group’s preference between X and Y also remains unchanged.
        \item There is no “dictator”: no single voter possesses the power to always determine the group’s preference.
    \end{itemize}
\end{itemize}

% RELATED WORK
    % How have other researchers tackled the problem
    % Outline benefits and downsides of other methods
    % Give a good structure to the current state of the art
\section{Related Work}\label{3RelatedWork}

\bibliographystyle{IEEtran}
\bibliography{refs}

\vspace{12pt}

\end{document}
